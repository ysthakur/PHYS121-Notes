\section{Chapter 5}

\subsection{Friction}

There are three kinds of friction:

\begin{itemize}
    \item Static friction: $\Vec{f}_s \leq \mu_s n$. Applies when object is not moving. It adjusts itself so that the net force is 0, but it can only increase to $\mu_s n$ at most.
    \item Kinetic friction: $\Vec{f}_k = \mu_k n$. Applies when object is moving.
    \item Rolling friction: $\Vec{f}_r = \mu_r n$. Applies when object is rolling.
\end{itemize}

\subsubsection{Interacting Objects}

According to Newton's Third Law:

\begin{itemize}
    \item Every force occurs as one member of an action/reaction pair of forces.
    \item The two forces act on \textbf{different} objects.
    \item The two forces point in opposite directions and have the same magnitude.
\end{itemize}

\subsection{Ropes and Pulleys}

\subsubsection{Ropes}

\textbf{Massless string approximation:} Mass of rope is 0.

Generally, \textbf{the tension in a massless string/rope equals the magnitude
    of the force pulling on the end of the string/rope}. As a result:
\begin{itemize}
    \item A massless string/rope ``transmits'' a force undiminished from one end
          to the other, i.e., if you pull on one end of a rope with force $F$,
          the other end pulls on whatever it's attached to with the same force.
    \item The tension in a massless string/rope is the same from one end to the
          other.
\end{itemize}

\subsubsection{Pulleys}

The tension in a massless string is unchanged by passing over a massless,
frictionless pulley (assume such a pulley for problems in Chapter 5).
